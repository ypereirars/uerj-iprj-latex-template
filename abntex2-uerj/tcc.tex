\documentclass{abntex2-uerj}

\usepackage{lipsum} % para geração de dummy text
\usepackage{tikz}
\usepackage{wrapfig}
\usetikzlibrary{calc}

\titulo{Título do Trabalho}
\autor{Yuri Pereira Ribeiro Silveira}
\local{Nova Friburgo}
\data{\today}

\orientador{Nome do Orientador}
\coorientador[Coorientador:]{<Nome do(a) Coorientador(a)>}

\instituicao{Universidade do Estado do Rio de Janeiro}
\tipotrabalho{monografia}

\naturezatrabalho{Este Trabalho de Conclusão de Curso foi julgado adequado à obtenção do título de... <Bacharel, Licenciado, Mestre, Especialização ou Tecnólogo> em... <área de concentração> e aprovado em sua forma final pelo Curso de... <nome do curso>, da \imprimirinstituicao.}
\preambulo{Trabalho de conclusão de curso apresentado como pré-requisito para obtenção do título de Engenheiro de Computação, ao Departamento de Modelagem Computacional, do Instituto Politécnico, da Universidade do Estado do Rio de Janeiro.}

\membrobancaA[\imprimirinstituicao]{Prof. e orientador \imprimirorientador}
\membrobancaB[Universidade B]{Membro B}
\membrobancaC[Universidade C]{Membro C}

\begin{document}

    % Elementos pré-textuais
    \pretextual
    %Capa
    \imprimircapa
    %Folha de rosto
    \imprimirfolhaderosto
    %Catalogação Biblioteca
    %\imprimircatalogacao
    %Folha de aprovação
    %\imprimirfolhaaprovacao
    %% Elemento opcional (Figura 10).
%% A palavra DEDICATÓRIA deve ser grafada em fonte 12, em
%% maiúsculas, negritada e centralizada na parte superior da folha.
%% O texto da dedicatória deve estar localizado na parte inferior da
%% folha, seguindo as regras gerais de apresentação gráfica.

\begin{epigrafe}[Dedicatória]

Dedico este trabalho à minha família, grandes responsáveis pela minha educação.

\end{epigrafe}
    %% A palavra AGRADECIMENTOS deve ser grafada em fonte 12,
%% em maiúsculas, negritada e centralizada na parte superior da folha.
%% O texto dos agradecimentos deve ser separado do título por duas
%% linhas em branco com espaçamento 1,5 e digitado de acordo as regras
%% gerais de apresentação gráfica.
%% Se houver necessidade, o texto pode continuar nas folhas seguin-
%% tes, sem incluir a palavra Agradecimentos.

\begin{agradecimentos}

\lipsum[1]

\end{agradecimentos}
    %% Elemento opcional.
%% É uma citação sem aspas – em fonte 12, estilo normal, com es-
%% paço 1,5 – seguida da indicação de autoria, grafada em fonte 12 e em
%% itálico.
%% O texto deve estar localizado no terço inferior da folha, com o ali-
%% nhamento livre, necessário à epígrafe.

\begin{epigrafe}

\noindent
Lorem ipsum dolor sit amet, consectetur adipiscing elit. Proin nibh est, facilisis et porttitor a, ultrices id tellus.\\
\hspace*{\fill} \textit{Nome Autor}

\end{epigrafe}
    %% 3.1.9 Resumo em língua portuguesa
%% Elemento obrigatório (Figura 14).
%% Consiste na apresentação sucinta dos pontos relevantes do texto,
%% em um único parágrafo. O resumo deve conter entre 150 e 500 pala-
%% vras e fornecer uma visão rápida e clara dos objetivos, da metodologia,
%% dos resultados e das conclusões do trabalho. Na elaboração do resumo,
%% deve-se usar o verbo na voz ativa, na terceira pessoa do singular.

%% Fonte -> TNR ou Arial, corpo 12.
%% A palavra RESUMO deve aparecer em letras maiúsculas
%% e em negrito.
%% O uso de itálico é permitido em palavras estrangeiras.
%% O uso de letras maiúsculas nas palavras-chave
%% restringe-se ao início da palavra, em nomes próprios
%% e siglas, se for o caso.

%% Alinhamento -> A palavra RESUMO deve estar localizada na margem
%% superior da folha e centralizada, e a referência, alinhada
%% à margem esquerda;
%% O alinhamento é justificado para o texto do resumo,
%% que inicia com parágrafo, e para as palavras-chave.

%% Espaçamento -> A palavra RESUMO deve ser separada da referência por
%% duas linhas em branco de 1,5;
%% Espaço 1 na referência e no resumo e, nas palavras-
%% chave, espaço 1,5.

%% Formato do papel,
%% orientação e margens -> Conforme especificado na seção 1.1.

%% Pontuação -> As palavras-chave devem ser separadas por ponto e
%% terminadas por ponto.

\begin{resumo}

\noindent
\entradaAutor{}. \textit{\imprimirtitulo}. 2015. \pageref{LastPage} f. Trabalho de Conclusão de Curso (Graduação em Engenharia de Computação) - Instituto Politécnico, Universidade do Estado do Rio de Janeiro, Nova Friburgo, 2015.
\vspace{\onelineskip}

\setlength{\parindent}{1.3cm}
\lipsum[1]

\vspace{\onelineskip}
\noindent Palavras-chave: Visão computacional. Machine learning. Detecção de objetos. Classificador de imagens.

\end{resumo}
    %% Elemento obrigatório (Figura 15).
%% Consiste em uma tradução do resumo em português para uma
%% língua estrangeira (em inglês, ABSTRACT; em espanhol, RESUMEN;
%% em francês, RÉSUMÉ), em um único parágrafo, seguido das palavras-
%% -chave representativas do conteúdo do trabalho, na língua estrangeira
%% escolhida.
%% O resumo em outra língua também é precedido pela referência
%% do trabalho, substituindo-se o título em português pelo título na língua
%% estrangeira adotada.
%% No caso de teses, é possível incluir dois resumos em língua es-
%% trangeira.
%% A apresentação gráfica e a ordem dos elementos seguem a mes-
%% ma orientação do resumo em português.

\begin{resumo}[Abstract]
\begin{otherlanguage*}{english}

\noindent
\entradaAutor{}. \textit{\englishTitle{}}. 2015. \pageref{LastPage} f. Trabalho de Conclusão de Curso (Graduação em Engenharia de Computação) - Instituto Politécnico, Universidade do Estado do Rio de Janeiro, Nova Friburgo, 2015.
\vspace{\onelineskip}

\setlength{\parindent}{1.3cm}
\lipsum[1]

\vspace{\onelineskip}
\noindent Keywords: Visão computacional. Machine learning. Detecção de objetos. Classificador de imagens.

\end{otherlanguage*}
\end{resumo}
    \listoffigures*
    \cleardoublepage
    \listoftables*
    \cleardoublepage
    \include{abreviaturas}
    \begin{siglas}
\item[UERJ] Universidade do Estado do Rio de Janeiro
\item[IPRJ] Instituto Politécnico do Rio de Janeiro
\item[HTTP] Hypertext Transfer Protocol
\end{siglas}
    \include{simbolos}
    \tableofcontents*
    \cleardoublepage

    % Elementos textuais
    \textual
    \include{cap1}
    \chapter{Cap. 2}

\lipsum

\lipsum[1]
    \include{cap3}
    \include{cap4}
    \chapter{Cap. 5}

\lipsum

    % Elementos pós-textuais
    \postextual
    \include{glossario}
    \include{apendices}
    \include{anexos}
    \bibliography{tcc}

\end{document}