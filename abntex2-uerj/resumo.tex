%% 3.1.9 Resumo em língua portuguesa
%% Elemento obrigatório (Figura 14).
%% Consiste na apresentação sucinta dos pontos relevantes do texto,
%% em um único parágrafo. O resumo deve conter entre 150 e 500 pala-
%% vras e fornecer uma visão rápida e clara dos objetivos, da metodologia,
%% dos resultados e das conclusões do trabalho. Na elaboração do resumo,
%% deve-se usar o verbo na voz ativa, na terceira pessoa do singular.

%% Fonte -> TNR ou Arial, corpo 12.
%% A palavra RESUMO deve aparecer em letras maiúsculas
%% e em negrito.
%% O uso de itálico é permitido em palavras estrangeiras.
%% O uso de letras maiúsculas nas palavras-chave
%% restringe-se ao início da palavra, em nomes próprios
%% e siglas, se for o caso.

%% Alinhamento -> A palavra RESUMO deve estar localizada na margem
%% superior da folha e centralizada, e a referência, alinhada
%% à margem esquerda;
%% O alinhamento é justificado para o texto do resumo,
%% que inicia com parágrafo, e para as palavras-chave.

%% Espaçamento -> A palavra RESUMO deve ser separada da referência por
%% duas linhas em branco de 1,5;
%% Espaço 1 na referência e no resumo e, nas palavras-
%% chave, espaço 1,5.

%% Formato do papel,
%% orientação e margens -> Conforme especificado na seção 1.1.

%% Pontuação -> As palavras-chave devem ser separadas por ponto e
%% terminadas por ponto.

\begin{resumo}

\noindent
PINTO, Marcos Belchior Monteiro Morete. \textit{Aplicação de técnicas de Visão Computacional, Machine Learning e SLAM a uma plataforma robótica para realização de tarefas de horticultura}. 2015. \pageref{LastPage} f. Trabalho de Conclusão de Curso (Graduação em Engenharia de Computação) - Instituto Politécnico, Universidade do Estado do Rio de Janeiro, Nova Friburgo, 2015.
\vspace{\onelineskip}

\setlength{\parindent}{1.3cm}
Este trabalho apresenta uma aplicação de métodos de visão computacional e Machine Learning em sistemas embutidos. A visão computacional assume atualmente um importante papel na execução de tarefas repetitivas de inspeção. O Machine Learning vêm sendo aplicado para reconhecimento de padrões em grupo de dados de natureza similar. A aplicação dos sistemas que utilizam visão computacional conjugado com o Machine Learning fundamenta-se  em diferentes áreas da computação moderna. No desenvolvimento do trabalho produziu-se um protótipo para automatização de processos de horticultura de pequeno porte utilizando a plataforma de prototipagem Arduino e o micro computador baseado em um system on a chip (SoC) Raspberry Pi. A combinação das técnicas computacionais e o desenvolvimento de um protótipo favoreceu na comprovação do funcionamento dos algoritmos empregados.		

\vspace{\onelineskip}
\noindent Palavras-chave: Visão computacional. Machine learning. Detecção de objetos. Viola Jones. Classificador de imagens. SLAM.

\end{resumo}