\documentclass[
    chapter=TITLE,
    12pt, 		% Tamanho da fonte
    openright, 	% Capítulos começam em pág. ímpar (insere página vazia caso preciso)
    oneside, 	% Para impressão somente verso. Oposto a impressão em verso e anverso
    a4paper, 	% Tamanho do papel
    english, 	% Idioma adicional para hifenização
    brazil 		% o último idioma é o principal do documento
    ]{abntex2}
    
\usepackage[utf8]{inputenc}
\usepackage{tikz}
\usepackage{wrapfig}
\usetikzlibrary{calc}
\usepackage{lipsum} 	% para geração de dummy text
\usepackage{multirow}

%% Redefine booleans
%\setboolean{ABNTEXuppersubsection}{false}

% Pacotes
%% Sumário
\usepackage{tocloft}

%% Codificação
\usepackage[T1]{fontenc} 	% Codificação de saída

%% Fonte
\usepackage{times} 			% Times New Roman
\renewcommand
	{\familydefault}
	{\sfdefault} 			% Seta fonte default

\usepackage{lastpage} 		% Usado pela Ficha catalográfica
\usepackage{indentfirst} 	% Indenta o primeiro parágrafo de cada seção.
\usepackage{color} 			% Controle das cores
\usepackage{microtype} 		% Para melhorias de justificação
\usepackage{lscape} 		% Permite a criação de conteúdo em modo paisagem
\usepackage{hyperref} 		% Criação de links
\usepackage{etoolbox}

%% Cabeçalho padrao
\makepagestyle
	{abntheadings}
\makeevenhead
	{abntheadings}
	{\ABNTEXfontereduzida\thepage}
	{}
	{\ABNTEXfontereduzida\textit\leftmark}
\makeoddhead
	{abntheadings}
	{}
	{}
	{\ABNTEXfontereduzida\thepage}
\makeheadrule
	{abntheadings}
	{0pt}
	{\normalrulethickness}
%%%%%%%%%%%%%%%%%%%%%%%%%%%%%%%%%%%%%%%%%%%%%%%%%%%%%%%%%%%%%%%%%


%% Cabeçalho do inicio do capitulo
\makepagestyle
	{abntchapfirst}
\makeoddhead
	{abntchapfirst}
	{}
	{}
	{\ABNTEXfontereduzida\thepage}

%% Tabelas
\usepackage{multirow}

%% Sumário
\usepackage{textcase}

%% Imagens
\usepackage{graphicx} % Inclusão de gráficos

%% Tamanho e alinhamento das legendas
\usepackage[
	singlelinecheck=false,
	justification=justified,
	font=footnotesize
	]
	{caption}

%% Citações
\usepackage[hyphenbreaks]{breakurl}
\usepackage[alf,abnt-emphasize=bf]{abntex2cite}

% Estilos
\autor{Nome SegundoNnome Sobrenome} 	%Nome do autor
% Especificidades para a entrada de autor pessoal: – sobrenome com indicativo de parentesco
% Quando o autor é brasileiro, trate o grau de parentesco como parte do sobrenome.
\newcommand{\entradaAutor}{SOBRENOME, N. S} % Sem ponto no final
% Titulo do trabalho
\titulo{Utilização do Latex na elaboração de trabalhos de conclusão de curso} %Título do trabalho
\newcommand{\englishTitle}{Preparation of Monographs with Latex}
\data{\today}			%Ano de publicação
\local{Nova Friburgo}	%Local
%% Membros da banca
\newcommand{\membrobancaA}{}
\newcommand{\membrobancaB}{}
\newcommand{\membrobancaC}{}

\providecommand
	{\imprimirmembrobancaA}
	{}
\providecommand
	{\imprimirmembrobancaAinst}
	{}
\renewcommand
	{\membrobancaA}
	[2]
	[\imprimirinstituicao]
	{\renewcommand
		{\imprimirmembrobancaA}
		{#2}
		\renewcommand
			{\imprimirmembrobancaAinst}
			{#1}
	}

\providecommand
	{\imprimirmembrobancaB}
	{}
\providecommand
	{\imprimirmembrobancaBinst}
	{}
\renewcommand
	{\membrobancaB}
	[2]
	[\imprimirinstituicao]
	{
		\renewcommand
			{\imprimirmembrobancaB}
			{#2}
			\renewcommand
				{\imprimirmembrobancaBinst}
				{#1}
	}

\providecommand
	{\imprimirmembrobancaC}
	{}
\providecommand
	{\imprimirmembrobancaCinst}
	{}
\renewcommand
	{\membrobancaC}
	[2]
	[\imprimirinstituicao]
	{
		\renewcommand
			{\imprimirmembrobancaC}
			{#2}
			\renewcommand
				{\imprimirmembrobancaCinst}
				{#1}
	}

\newcommand
	{\imprimirmeOrientadorinst}
	{\imprimirinstituicao}
\newcommand
	{\imprimirmeCoorientadorinst}
	{\imprimirinstituicao}

%% Natureza do trabalho
\providecommand{\imprimirnaturezatrabalho}{}
\newcommand{\naturezatrabalho}[1]{
    \renewcommand{\imprimirnaturezatrabalho}{#1}
}

%% Redefinir resumo
\renewenvironment{resumo}[1][\resumoname]{%
   \pretextualchapter{#1}
  }{\PRIVATEclearpageifneeded}
  
%% Redefinir dedicatória
\renewenvironment{dedicatoria}[1][]
	{
	     \ifthenelse{\equal{#1}{}}{
	             \PRIVATEbookmarkthis{\dedicatorianame}
	     }{\preamblealchapter{#1}}
	
	     \vspace*{\fill}
	}
	{}

\renewenvironment{epigrafe}[1][]
{
        \ifthenelse{\equal{#1}{}}{
                \PRIVATEbookmarkthis{\epigraphname}
        }{\pretextualchapter{#1}}

        \vspace*{\fill}
}
{

}

%% Comando para inserir sigla
\newcommand{\sigla}[2][]{
    \item[#1] \textit{#2}
}

%% Redefinição da formatação do \chapter
\renewcommand{\ABNTEXchapterfont}{\normalfont\fontseries{b}\selectfont}
\renewcommand{\ABNTEXchapterfontsize}{\normalsize}
\renewcommand{\ABNTEXpartfont}{\fontseries{b}\selectfont\selectfont}
\renewcommand{\ABNTEXpartfontsize}{\normalsize}
\renewcommand{\ABNTEXsectionfont}{\normalfont\selectfont} %\fontseries{b}
\renewcommand{\ABNTEXsectionfontsize}{\normalsize}
\renewcommand{\ABNTEXsubsectionfont}{\normalfont}
\renewcommand{\ABNTEXsubsectionfontsize}{\normalsize}
\renewcommand{\ABNTEXsubsubsectionfont}{\normalfont}
\renewcommand{\ABNTEXsubsubsectionfontsize}{\normalsize}
\renewcommand{\ABNTEXsubsubsubsectionfont}{\normalfont}
\renewcommand{\ABNTEXsubsubsubsectionfontsize}{\normalsize}
%\renewcommand{\ABNTEXsubsubsectionfont}{\normalfont\fontseries{b}\selectfont}
%\renewcommand{\ABNTEXsubsubsectionfontsize}{\normalsize}
%\renewcommand{\ABNTEXsubsubsubsectionfont}{\normalfont\itshape\selectfont}
%\renewcommand{\ABNTEXsubsubsubsectionfontsize}{\normalsize}



%% Redefinição da formatação de Parágrafos 
\setlength{\parindent}{1.3cm}
\setlength{\parskip}{0cm}

%% Sumário
%\renewcommand{\cftchapterdotsepfont}{\normalfont\normalsize}
\renewcommand*{\cftchapterfont}{\normalfont\fontseries{b}} %\fontseries{b}
\renewcommand{\chapnumfont}{\normalfont\normalsize}
\renewcommand*{\cftsectionfont}{\normalfont} %\fontseries{b}
\renewcommand*{\cftsubsectionfont}{\normalfont}
\renewcommand*{\cftsubsubsectionfont}{\normalfont}
\renewcommand*{\cftsubsubsubsectionfont}{\normalfont}
\renewcommand*{\cftparagraphfont}{\normalfont} %\itshape

%% Forca underline na secao terciaria
\newcommand{\tmpsubsection}[1]{}
\let\tmpsubsection=\subsection
\renewcommand{\subsection}[1]{\tmpsubsection{\underline{#1}}}

%% Forca underline na secao secundaroa
\newcommand{\tmpsection}[1]{}
\let\tmpsection=\section
\renewcommand{\section}[1]{\tmpsection{\textbf{#1}}}

%% Tornar as seções secundários com fonte em maiúscula
%\makeatletter
%\let\oldcontentsline\contentsline
%\def\contentsline#1#2{%
%  \expandafter\ifx\csname l@#1\endcsname\l@section
%    \expandafter\@firstoftwo
%  \else
%    \expandafter\@secondoftwo
%  \fi
%  {%
%    \oldcontentsline{#1}{\MakeTextUppercase{#2}}%
%  }{%
%    \oldcontentsline{#1}{#2}%
%  }%
%}
%\makeatother

%% Tornar a entrada "Referências" com fonte em maiúscula
%%\addto\captionsbrazil{
%%    \renewcommand{\bibname}{REFER\^ENCIAS}
%%}

% Legendas
\makeatletter
\patchcmd{\caption@@@make}
  {\ifcaption@star}
    {\ifcaption@star\small}
    {}{}
\makeatother


\graphicspath{ {figures/} }


\preambulo{
	Trabalho de conclusão de curso apresentado como pré-requisito para obtenção do título de Engenheiro de Computação, ao Departamento de Modelagem Computacional, do Instituto Politécnico, da Universidade do Estado do Rio de Janeiro.
}
\orientador{Nome do Orientador}
\tipotrabalho{monografia}

\begin{document}
	%Capa
	\renewcommand{\imprimircapa}{%
	\begin{tikzpicture}[remember picture, overlay]
		\draw[line width = 4pt] ($(current page.north west) + (20mm, -20mm)$) rectangle ($(current page.south east) + (-10mm,10mm)$);
		\draw[line width = 1pt] ($(current page.north west) + (21.5mm,-21.5mm)$) rectangle ($(current page.south east) + (-11.5mm,11.5mm)$);
	\end{tikzpicture}
	\begin{capa}
		\begin{minipage}[r]{0.1\linewidth}
			\includegraphics[width=1.5\linewidth]{uerj} 
		\end{minipage}%
		\begin{minipage}[c]{0.7\linewidth}
			\begin{center}
				\textbf{UNIVERSIDADE DO ESTADO DO \\ RIO DE JANEIRO}\\
				\noindent\rule{7cm}{0.4pt}\\
				\textbf{INSTITUTO POLITÉCNICO \\ GRADUAÇÃO EM ENGENHARIA \\DE COMPUTAÇÃO}\\
			\end{center}
		\end{minipage}%
		\begin{minipage}[c]{0.1\linewidth}
			\includegraphics[width=1.5\linewidth]{iprj} 
		\end{minipage}
		
		\vspace*{4.5cm}
		\center
		\textbf{\large\imprimirautor}
		\vspace*{3cm}
		\begin{center}
			\bfseries\LARGE\imprimirtitulo
		\end{center}
		\vfill
		\textbf{
			\large\imprimirlocal
			\\	
			\large \the\year
		}
		\vspace*{1cm}
	\end{capa}
}
\imprimircapa
	
	%Folha de rosto
	\makeatletter
\renewcommand{\folhaderostocontent}{
	\begin{tikzpicture}[remember picture, overlay]
	\draw[line width = 4pt] ($(current page.north west) + (20mm, -20mm)$) rectangle ($(current page.south east) + (-10mm,10mm)$);
	\draw[line width = 1pt] ($(current page.north west) + (21mm,-21mm)$) rectangle ($(current page.south east) + (-11.5mm,11.5mm)$);
	\end{tikzpicture}
	
	\begin{minipage}[r]{0.1\linewidth}
		\includegraphics[width=1.5\linewidth]{uerj} 
	\end{minipage}%
	\begin{minipage}[c]{0.7\linewidth}
		\begin{center}
			\textbf{UNIVERSIDADE DO ESTADO DO \\ RIO DE JANEIRO}\\
			\noindent\rule{7cm}{0.4pt}\\
			\textbf{INSTITUTO POLITÉCNICO \\ GRADUAÇÃO EM ENGENHARIA \\DE COMPUTAÇÃO}\\
		\end{center}
	\end{minipage}%
	\begin{minipage}[c]{0.1\linewidth}
		\includegraphics[width=1.5\linewidth]{iprj} 
	\end{minipage}
	\vspace*{2cm}
	\begin{center}
		{\large\imprimirautor}
		\vspace*{\fill}\vspace*{\fill}
		\begin{center}
			\bfseries\Large\imprimirtitulo
		\end{center}
		\vspace*{\fill}
		\abntex@ifnotempty{\imprimirpreambulo}{%
			\hspace{.45\textwidth}
			\begin{minipage}{.5\textwidth}
				\SingleSpacing
				\imprimirpreambulo
			\end{minipage}%
			\vspace*{2cm}
			\vspace*{\fill}
		}%
		{\abntex@ifnotempty{\imprimirinstituicao}{\imprimirinstituicao
				\vspace*{\fill}}}
		{\large\imprimirorientadorRotulo~\imprimirorientador\par}
		\abntex@ifnotempty{\imprimircoorientador}{%
			{\large\imprimircoorientadorRotulo~\imprimircoorientador}%
		}%
		\vspace*{\fill}
		{\large\imprimirlocal}
		\par
		{\large\the\year}
		\vspace*{1cm}
	\end{center}
}
\makeatother
	\imprimirfolhaderosto
	
	%Catalogação Biblioteca
	\newcommand{\imprimircatalogacao}{%
	\noindent
	UNIVERSIDADE DO ESTADO DO RIO DE JANEIRO\\
	INSTITUTO POLITÉCNICO\\
	CURSO DE ENGENHARIA DE COMPUTAÇÃO\\\\
	
	\noindent
	Reitor: Ruy Garcia Marques\\
	Vice-Reitor: Maria Georgina Muniz Washington\\
	Diretor do Instituto Politécnico: Ricardo Carvalho de Barros\\
	Coordenador de Curso: José Humberto Zani\\
	
	\noindent
	Banca Avaliadora Composta por: \imprimirorientador~(Orientador)\\
	Prof. Dr. Professor 1\\
	Prof. Dr. Professor 2
	
	\begin{center}
		CATALOGAÇÃO NA FONTE \\ UERJ/REDE SIRIUS/BIBLIOTECA CTC/E
		\begin{tikzpicture}[remember picture, overlay]
			\draw[line width = 1pt] ($(current page.north west) + (55mm, -111.5mm)$) rectangle ($(current page.south east) + (-35.5mm,80mm)$);
		\end{tikzpicture}
	\end{center}
	
	Endereço: UERJ - IPRJ \\
			  Rua Bonfim, 25 - Prédio 5, Vila Amélia.\\
			  CEP 28625-570 - Nova Friburgo - RJ - Brasil.\\

	\noindent
	Este trabalho nos termos da legislação que resguarda os direitos autorais é considerado de propriedade da Universidade do Estado do Rio de Janeiro (UERJ). É permitida a transcrição parcial de partes do trabalho, ou mencioná-lo, para comentários e citações, desde que sem propósitos comerciais e que seja feita a referência bibliográfica completa.
	
	\begin{flushright}
		\assinatura{\imprimirautor}
	\end{flushright}
}

\imprimircatalogacao
	
	%Folha de aprovação
	\newcommand{\imprimirfolhaaprovacao}{%
	\begin{folhadeaprovacao}
		\begin{center}
			{\large{\imprimirautor}}\\[1cm]
			\vspace{2cm}
			{\Large\bfseries\imprimirtitulo}
		\end{center}		
		\vspace{1cm}
		\hspace{.45\textwidth}
		\begin{minipage}{.5\textwidth}
			\imprimirpreambulo
		\end{minipage}%
		\\\\\\\\
		\vspace{1cm}
		Aprovado em \imprimirdata.\\
		\vspace{1.5cm}
		Banca examinadora:
		
		\assinatura{\imprimirorientador\\Instituto Politécnico - UERJ}
		\assinatura{Professor da banca 1\\Instituto Politécnico - UERJ}
		\assinatura{Professor da banca 2\\Instituto Politécnico - UERJ}
		\begin{center}
			\vfill
			{\large\imprimirlocal}
			\par
			{\large\the\year}
		\end{center}
		
	\end{folhadeaprovacao}
}

\imprimirfolhaaprovacao

	%% Elemento opcional (Figura 10).
%% A palavra DEDICATÓRIA deve ser grafada em fonte 12, em
%% maiúsculas, negritada e centralizada na parte superior da folha.
%% O texto da dedicatória deve estar localizado na parte inferior da
%% folha, seguindo as regras gerais de apresentação gráfica.

\begin{epigrafe}[Dedicatória]

Dedico este trabalho à minha família, grandes responsáveis pela minha educação.

\end{epigrafe}
	%% A palavra AGRADECIMENTOS deve ser grafada em fonte 12,
%% em maiúsculas, negritada e centralizada na parte superior da folha.
%% O texto dos agradecimentos deve ser separado do título por duas
%% linhas em branco com espaçamento 1,5 e digitado de acordo as regras
%% gerais de apresentação gráfica.
%% Se houver necessidade, o texto pode continuar nas folhas seguin-
%% tes, sem incluir a palavra Agradecimentos.

\begin{agradecimentos}

\lipsum[1]

\end{agradecimentos}
	%% Elemento opcional.
%% É uma citação sem aspas – em fonte 12, estilo normal, com es-
%% paço 1,5 – seguida da indicação de autoria, grafada em fonte 12 e em
%% itálico.
%% O texto deve estar localizado no terço inferior da folha, com o ali-
%% nhamento livre, necessário à epígrafe.

\begin{epigrafe}

\noindent
Lorem ipsum dolor sit amet, consectetur adipiscing elit. Proin nibh est, facilisis et porttitor a, ultrices id tellus.\\
\hspace*{\fill} \textit{Nome Autor}

\end{epigrafe}
	%% 3.1.9 Resumo em língua portuguesa
%% Elemento obrigatório (Figura 14).
%% Consiste na apresentação sucinta dos pontos relevantes do texto,
%% em um único parágrafo. O resumo deve conter entre 150 e 500 pala-
%% vras e fornecer uma visão rápida e clara dos objetivos, da metodologia,
%% dos resultados e das conclusões do trabalho. Na elaboração do resumo,
%% deve-se usar o verbo na voz ativa, na terceira pessoa do singular.

%% Fonte -> TNR ou Arial, corpo 12.
%% A palavra RESUMO deve aparecer em letras maiúsculas
%% e em negrito.
%% O uso de itálico é permitido em palavras estrangeiras.
%% O uso de letras maiúsculas nas palavras-chave
%% restringe-se ao início da palavra, em nomes próprios
%% e siglas, se for o caso.

%% Alinhamento -> A palavra RESUMO deve estar localizada na margem
%% superior da folha e centralizada, e a referência, alinhada
%% à margem esquerda;
%% O alinhamento é justificado para o texto do resumo,
%% que inicia com parágrafo, e para as palavras-chave.

%% Espaçamento -> A palavra RESUMO deve ser separada da referência por
%% duas linhas em branco de 1,5;
%% Espaço 1 na referência e no resumo e, nas palavras-
%% chave, espaço 1,5.

%% Formato do papel,
%% orientação e margens -> Conforme especificado na seção 1.1.

%% Pontuação -> As palavras-chave devem ser separadas por ponto e
%% terminadas por ponto.

\begin{resumo}

\noindent
\entradaAutor{}. \textit{\imprimirtitulo}. 2015. \pageref{LastPage} f. Trabalho de Conclusão de Curso (Graduação em Engenharia de Computação) - Instituto Politécnico, Universidade do Estado do Rio de Janeiro, Nova Friburgo, 2015.
\vspace{\onelineskip}

\setlength{\parindent}{1.3cm}
\lipsum[1]

\vspace{\onelineskip}
\noindent Palavras-chave: Visão computacional. Machine learning. Detecção de objetos. Classificador de imagens.

\end{resumo}
	%% Elemento obrigatório (Figura 15).
%% Consiste em uma tradução do resumo em português para uma
%% língua estrangeira (em inglês, ABSTRACT; em espanhol, RESUMEN;
%% em francês, RÉSUMÉ), em um único parágrafo, seguido das palavras-
%% -chave representativas do conteúdo do trabalho, na língua estrangeira
%% escolhida.
%% O resumo em outra língua também é precedido pela referência
%% do trabalho, substituindo-se o título em português pelo título na língua
%% estrangeira adotada.
%% No caso de teses, é possível incluir dois resumos em língua es-
%% trangeira.
%% A apresentação gráfica e a ordem dos elementos seguem a mes-
%% ma orientação do resumo em português.

\begin{resumo}[Abstract]
\begin{otherlanguage*}{english}

\noindent
\entradaAutor{}. \textit{\englishTitle{}}. 2015. \pageref{LastPage} f. Trabalho de Conclusão de Curso (Graduação em Engenharia de Computação) - Instituto Politécnico, Universidade do Estado do Rio de Janeiro, Nova Friburgo, 2015.
\vspace{\onelineskip}

\setlength{\parindent}{1.3cm}
\lipsum[1]

\vspace{\onelineskip}
\noindent Keywords: Visão computacional. Machine learning. Detecção de objetos. Classificador de imagens.

\end{otherlanguage*}
\end{resumo}
	\listoffigures*
	\cleardoublepage
	\listoftables*
	\cleardoublepage
	%\include{abreviaturas}
	\begin{siglas}
\item[UERJ] Universidade do Estado do Rio de Janeiro
\item[IPRJ] Instituto Politécnico do Rio de Janeiro
\item[HTTP] Hypertext Transfer Protocol
\end{siglas}
	\tableofcontents*
	\cleardoublepage
	
	 % Elementos textuais
	\textual
	\include{cap1}
	\chapter{Cap. 2}

\lipsum

\lipsum[1]
	\include{cap3}
	\include{cap4}
	\chapter{Cap. 5}

\lipsum

	% Elementos pós-textuais
	\postextual
	%\include{glossario}
	\include{apendices}
	\include{anexos}
	\bibliography{tcc}
\end{document}