%% Elemento obrigatório (Figura 15).
%% Consiste em uma tradução do resumo em português para uma
%% língua estrangeira (em inglês, ABSTRACT; em espanhol, RESUMEN;
%% em francês, RÉSUMÉ), em um único parágrafo, seguido das palavras-
%% -chave representativas do conteúdo do trabalho, na língua estrangeira
%% escolhida.
%% O resumo em outra língua também é precedido pela referência
%% do trabalho, substituindo-se o título em português pelo título na língua
%% estrangeira adotada.
%% No caso de teses, é possível incluir dois resumos em língua es-
%% trangeira.
%% A apresentação gráfica e a ordem dos elementos seguem a mes-
%% ma orientação do resumo em português.

\begin{resumo}[Abstract]
\begin{otherlanguage*}{english}

\noindent
PINTO, Marcos Belchior Monteiro Morete. \textit{Application of computer vision, Machine Learning and SLAM a robotic platform for realization of tasks of horticulture}. 2015. \pageref{LastPage} f. Trabalho de Conclusão de Curso (Graduação em Engenharia de Computação) - Instituto Politécnico, Universidade do Estado do Rio de Janeiro, Nova Friburgo, 2015.
\vspace{\onelineskip}

\setlength{\parindent}{1.3cm}
This work presents an application of computer vision and machine learning methods in embedded systems. Computer vision currently plays an important role in execution of repetitive inspection tasks. Machine learning has been applied to recognize patterns in a similar data set. The application of computer vision methods with machine learning algorithms is supported by different areas of modern computing. During the development, A prototype for automation of simple horticultural processes was constructed using the Arduino prototyping platform and the Raspberry Pi microcomputer system on a chip (SoC). The combination of computing techniques and the development of a prototype helped to check the correctness and convenience of the algorithms applied.

\vspace{\onelineskip}
\noindent Keywords: Computer vision. Machine learning. Object detection. Viola Jones. Image classifier. SLAM.

\end{otherlanguage*}
\end{resumo}