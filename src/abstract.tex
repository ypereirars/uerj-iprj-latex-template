%% Elemento obrigatório (Figura 15).
%% Consiste em uma tradução do resumo em português para uma
%% língua estrangeira (em inglês, ABSTRACT; em espanhol, RESUMEN;
%% em francês, RÉSUMÉ), em um único parágrafo, seguido das palavras-
%% -chave representativas do conteúdo do trabalho, na língua estrangeira
%% escolhida.
%% O resumo em outra língua também é precedido pela referência
%% do trabalho, substituindo-se o título em português pelo título na língua
%% estrangeira adotada.
%% No caso de teses, é possível incluir dois resumos em língua es-
%% trangeira.
%% A apresentação gráfica e a ordem dos elementos seguem a mes-
%% ma orientação do resumo em português.

\begin{resumo}[Abstract]
\begin{otherlanguage*}{english}

\noindent
\entradaAutor{}. \textit{\englishTitle{}}. 2015. \pageref{LastPage} f. Trabalho de Conclusão de Curso (Graduação em Engenharia de Computação) - Instituto Politécnico, Universidade do Estado do Rio de Janeiro, Nova Friburgo, 2015.
\vspace{\onelineskip}

\setlength{\parindent}{1.3cm}
\lipsum[1]

\vspace{\onelineskip}
\noindent Keywords: Visão computacional. Machine learning. Detecção de objetos. Classificador de imagens.

\end{otherlanguage*}
\end{resumo}