\documentclass[oneside,12pt]{abntex2}
\usepackage[utf8]{inputenc}
\usepackage{tikz}
\usepackage{wrapfig}
\usetikzlibrary{calc}

\graphicspath{ {figures/} }

\RequirePackage{times} % Times New Roman
\renewcommand{\familydefault}{\sfdefault} % Seta fonte default
\renewcommand{\ABNTEXchapterfont}{\normalfont\fontseries{b}\selectfont}
\renewcommand{\ABNTEXchapterfontsize}{\normalsize}
\newcommand{\ABNTEXchaptertwfont}{\sffamily}
\newcommand{\ABNTEXchaptertwfontsize}{\Huge}


\newcommand{\imprimircatalogacao}{%
	\noindent
	\ABNTEXchaptertwfont UNIVERSIDADE DO ESTADO DO RIO DE JANEIRO\\
	\ABNTEXchaptertwfont INSTITUTO POLITÉCNICO\\
	\ABNTEXchaptertwfont CURSO DE ENGENHARIA DE COMPUTAÇÃO\\\\
	
	\noindent
	Reitor: Ruy Garcia Marques\\
	Vice-Reitor: Maria Georgina Muniz Washington\\
	Diretor do Instituto Politécnico: Ricardo Carvalho de Barros\\
	Coordenador de Curso: José Humberto Zani\\
	
	\noindent
	Banca Avaliadora Composta por: \imprimirorientador~(Orientador)\\
	Prof. Dr. Professor 1\\
	Prof. Dr. Professor 2
	
	\begin{center}
		CATALOGAÇÃO NA FONTE \\ UERJ/REDE SIRIUS/BIBLIOTECA CTC/E
		\begin{tikzpicture}[remember picture, overlay]
			\draw[line width = 1pt] ($(current page.north west) + (55mm, -111.5mm)$) rectangle ($(current page.south east) + (-35.5mm,80mm)$);
		\end{tikzpicture}
	\end{center}
	
	Endereço: UERJ - IPRJ \\
			  Rua Bonfim, 25 - Prédio 5, Vila Amélia.\\
			  CEP 28625-570 - Nova Friburgo - RJ - Brasil.\\

	\noindent
	Este trabalho nos termos da legislação que resguarda os direitos autorais é considerado de propriedade da Universidade do Estado do Rio de Janeiro (UERJ). É permitida a transcrição parcial de partes do trabalho, ou mencioná-lo, para comentários e citações, desde que sem propósitos comerciais e que seja feita a referência bibliográfica completa.
	
	\begin{flushright}
		\assinatura{\imprimirautor}
	\end{flushright}
}

\newcommand{\imprimirfolhaaprovacao}{%
	\begin{folhadeaprovacao}
		\begin{center}
			{\ABNTEXchaptertwfont\large{\imprimirautor}}\\[1cm]
			\vspace{2cm}
			{\ABNTEXchapterfont\Large\bfseries\imprimirtitulo}
		\end{center}		
		\vspace{1cm}
		\hspace{.45\textwidth}
		\begin{minipage}{.5\textwidth}
			\imprimirpreambulo
		\end{minipage}%
		\\\\\\\\
		\vspace{1cm}
		Aprovado em \imprimirdata.\\
		\vspace{1.5cm}
		Banca examinadora:
		
		\assinatura{\imprimirorientador\\Instituto Politécnico - UERJ}
		\assinatura{Professor da banca 1\\Instituto Politécnico - UERJ}
		\assinatura{Professor da banca 2\\Instituto Politécnico - UERJ}
		\begin{center}
			\vfill
			{\large\imprimirlocal}
			\par
			{\large\the\year}
		\end{center}
		
	\end{folhadeaprovacao}
}

\renewcommand{\imprimircapa}{%
	\begin{tikzpicture}[remember picture, overlay]
		\draw[line width = 4pt] ($(current page.north west) + (20mm, -20mm)$) rectangle ($(current page.south east) + (-10mm,10mm)$);
		\draw[line width = 1pt] ($(current page.north west) + (21mm,-21mm)$) rectangle ($(current page.south east) + (-11.5mm,11.5mm)$);
	\end{tikzpicture}
	\begin{capa}
		\begin{minipage}[r]{0.1\linewidth}
			\includegraphics[width=1.5\linewidth]{uerj} 
		\end{minipage}%
		\begin{minipage}[c]{0.7\linewidth}
			\begin{center}
				\ABNTEXchaptertwfont \textbf{UNIVERSIDADE DO ESTADO DO \\ RIO DE JANEIRO}\\
				\noindent\rule{7cm}{0.4pt}\\
				\ABNTEXchaptertwfont \textbf{INSTITUTO POLITÉCNICO \\ GRADUAÇÃO EM ENGENHARIA \\DE COMPUTAÇÃO}\\
			\end{center}
		\end{minipage}%
		\begin{minipage}[c]{0.1\linewidth}
			\includegraphics[width=1.5\linewidth]{iprj} 
		\end{minipage}
		
		\vspace*{4.5cm}
		\center
		\textbf{\ABNTEXchaptertwfont\large\imprimirautor}
		\vspace*{3cm}
		\begin{center}
			\ABNTEXchaptertwfont\bfseries\LARGE\imprimirtitulo
		\end{center}
		\vfill
		\textbf{
			\large\imprimirlocal
			\\	
			\large \the\year
		}
		\vspace*{1cm}
	\end{capa}
}


\makeatletter
\renewcommand{\folhaderostocontent}{
	\begin{tikzpicture}[remember picture, overlay]
	\draw[line width = 4pt] ($(current page.north west) + (20mm, -20mm)$) rectangle ($(current page.south east) + (-10mm,10mm)$);
	\draw[line width = 1pt] ($(current page.north west) + (21mm,-21mm)$) rectangle ($(current page.south east) + (-11.5mm,11.5mm)$);
	\end{tikzpicture}
	
	\begin{minipage}[r]{0.1\linewidth}
		\includegraphics[width=1.5\linewidth]{uerj} 
	\end{minipage}%
	\begin{minipage}[c]{0.7\linewidth}
		\begin{center}
			\ABNTEXchaptertwfont \textbf{UNIVERSIDADE DO ESTADO DO \\ RIO DE JANEIRO}\\
			\noindent\rule{7cm}{0.4pt}\\
			\ABNTEXchaptertwfont \textbf{INSTITUTO POLITÉCNICO \\ GRADUAÇÃO EM ENGENHARIA \\DE COMPUTAÇÃO}\\
		\end{center}
	\end{minipage}%
	\begin{minipage}[c]{0.1\linewidth}
		\includegraphics[width=1.5\linewidth]{iprj} 
	\end{minipage}
	\vspace*{2cm}
	\begin{center}
		{\ABNTEXchaptertwfont\large\imprimirautor}
		\vspace*{\fill}\vspace*{\fill}
		\begin{center}
			\ABNTEXchaptertwfont\bfseries\Large\imprimirtitulo
		\end{center}
		\vspace*{\fill}
		\abntex@ifnotempty{\imprimirpreambulo}{%
			\hspace{.45\textwidth}
			\begin{minipage}{.5\textwidth}
				\SingleSpacing
				\imprimirpreambulo
			\end{minipage}%
			\vspace*{2cm}
			\vspace*{\fill}
		}%
		{\abntex@ifnotempty{\imprimirinstituicao}{\imprimirinstituicao
				\vspace*{\fill}}}
		{\large\imprimirorientadorRotulo~\imprimirorientador\par}
		\abntex@ifnotempty{\imprimircoorientador}{%
			{\large\imprimircoorientadorRotulo~\imprimircoorientador}%
		}%
		\vspace*{\fill}
		{\large\imprimirlocal}
		\par
		{\large\the\year}
		\vspace*{1cm}
	\end{center}
}
\makeatother

\autor{Yuri Pereira Ribeiro Silveira} 	%Nome do autor
\titulo{Título do Trabalho} 			%Título do trabalho
\data{\today}						%Ano de publicação
\local{Nova Friburgo}					%Local

\preambulo{
	Trabalho de conclusão de curso apresentado como pré-requisito para obtenção do título de Engenheiro de Computação, ao Departamento de Modelagem Computacional, do Instituto Politécnico, da Universidade do Estado do Rio de Janeiro.
}
\orientador{Nome do Orientador}
\tipotrabalho{monografia}

\begin{document}
	%Capa
	\imprimircapa
	
	%Folha de rosto
	\imprimirfolhaderosto
	
	%Catalogação Biblioteca
	\imprimircatalogacao
	
	%Folha de aprovação
	\imprimirfolhaaprovacao
	
	%Dedicatória
	\begin{dedicatoria}
		\vspace*{\fill}
		Lorem ipsum dolor sit amet, consectetur adipiscing elit, sed do eiusmod tempor incididunt ut labore et dolore magna aliqua. Ut enim ad minim veniam, quis nostrud exercitation ullamco laboris nisi ut aliquip ex ea commodo consequat. Duis aute irure dolor in reprehenderit in voluptate velit esse cillum dolore eu fugiat nulla pariatur. Excepteur sint occaecat cupidatat non proident, sunt in culpa qui officia deserunt mollit anim id est laborum.
		\vspace*{\fill}
	\end{dedicatoria}

	\begin{agradecimentos}
		Os agradecimentos...
	\end{agradecimentos}


	\begin{epigrafe}
		\vspace*{\fill}
		\begin{flushright}
			\textit{``Frase''\\
				(autor)}
		\end{flushright}
	\end{epigrafe}
	
	%Resumo em língua vernácula
	\begin{resumo}[RESUMO]
	\vspace{\onelineskip}

		PINTO, Marcos Belchior Monteiro Morete. Aplicação de técnicas de Visão Computacional, Machine Learning e SLAM a uma plataforma robótica para realização de tarefas de horticultura.  2015. 55 f. Trabalho de Conclusão de Curso (Graduação em Engenharia de Computação) - Instituto Politécnico, Universidade do Estado do Rio de Janeiro, Nova Friburgo, 2015.\\
		
		\\\setlength{\parindent}{1cm}Este trabalho apresenta uma aplicação de métodos de visão computacional e Machine Learning em sistemas embutidos. A visão computacional assume atualmente um importante papel na execução de tarefas repetitivas de inspeção. O Machine Learning vêm sendo aplicado para reconhecimento de padrões em grupo de dados de natureza similar. A aplicação dos sistemas que utilizam visão computacional conjugado com o Machine Learning fundamenta-se  em diferentes áreas da computação moderna. No desenvolvimento do trabalho produziu-se um protótipo para automatização de processos de horticultura de pequeno porte utilizando a plataforma de prototipagem Arduino e o micro computador baseado em um system on a chip (SoC) Raspberry Pi. A combinação das técnicas computacionais e o desenvolvimento de um protótipo favoreceu na comprovação do funcionamento dos algoritmos empregados.		
		\vspace{\onelineskip}
		\\\noindent
		Palavras-chave: palavras, chave.
	\end{resumo}

	%Resumo em língua estrangeira
	\begin{resumo}[Abstract]
		\begin{otherlanguage*}{english}
			Abstract in English
			\vspace{\onelineskip}
			\\\noindent
			\textbf{Keywords}: key, words.
		\end{otherlanguage*}
	\end{resumo}



	%Lista de siglas e abreviaturas
	\begin{siglas}
		\item[ABC] Letras
		\item[123] Números
	\end{siglas}


	%Lista de Figuras
	\pdfbookmark[0]{\listfigurename}{lof}
	\listoffigures*
	\cleardoublepage

	
	%Lista de tabelas
	\pdfbookmark[0]{\listtablename}{lot}
	\listoftables*
	\cleardoublepage
	
	%Sumário
	\pdfbookmark[0]{\contentsname}{toc}
	\tableofcontents*
	\cleardoublepage

	
	\chapter{INTRODUÇÃO}
		\section{Seção}
		\section{Seção}
		\section{Seção}
	\chapter{Desenvolvimento}
		\section{Seção}
			\subsection{}
		\section{Seção}
		\section{Seção}
	\chapter{Conclusão}
	\section{Seção}
	\subsection{}
		\section{Seção}
		\section{Seção}
			\subsection{Subseção}
				\subsubsection{Sub Sub Seção}
	\chapter{Referências}
\end{document}